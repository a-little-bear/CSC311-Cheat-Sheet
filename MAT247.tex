\documentclass[10pt]{LatexTemplate/hw}
\useclspackage{mat247}

\begin{document}

\begin{multicols*}{2}
Change of basis: If say $A$ is a linear transformation w.r.t. the standard basis, and $\be$ is a basis, then let $P$ be the matrix with columns as the basis vectors of $\be$, we have $[A]_{\be} = P^{-1}AP$.\\
Diagonalizable: if $A$ is similar to a diagonal matrix (with eigenvalues on the diagonal).

Direct Sum: $V=V_1\oplus...\oplus V_r$ if $v\in V$ can be uniquely written as $v=v_1+...+v_r$ where $v_i\in V_i$.\\
Sum is direct if and ony if $U_i\cap(\sum_{j\neq i}U_j) = \{0\}$ for all $i$.

$T$-invariant: $T$ is $V$-invariant if $T(V)\subseteq V$.\\
$p_T(z)$ is divisible by $p_{T|_W}(z)$, for $T$-invariant subspace $W$.\\
Diagonalizable if and only if $V$ has a basis of eigenvectors of $T$, if and only if it is a direct sum of eigenspaces of $T$.\\
$\ker(T)=(\ker(T)\cap V_1)\oplus...\oplus(\ker(T)\cap V_r)$\\
$\ran(T)=(\ran(T)\cap V_1)\oplus...\oplus(\ran(T)\cap V_r)$

$v\in V$ cyclic if $W=\operatorname{span}\{v,T(v),T^2(v),...\}=V$.\\
Let $W$ be cyclic subspace generated by $v$. Exists $k$ s.t. $v,T(v),...,T^{k-1}(v)$ is basis of $W$.

Given $p(z)=a_0+...+z^n$, $$A=\begin{pmatrix}
0 & 0 & \cdots & 0 & -a_0\\
1 & 0 & \cdots & 0 & -a_1\\
\cdots & \cdots & \cdots & \cdots & \cdots\\
0 & 0 & \cdots & 1 & -a_{n-1}
\end{pmatrix}$$ is the companion matrix of $p(z)$.\\
$p_A(z)=\det(zI-A)=p(z)$.\\
Matrix admits cyclic vector if and only if similar to a companion matrix.

Largest $T$-cyclic subspace admits $T$-invariant complement.\\
Finite dim $V$ can be decomposed into direct sum of $T$-cyclic subspaces by above.

Jordan block of size $k$ with type $\la$: $$J(\la, k)=\begin{pmatrix}
\la & 1 & 0 & \cdots & 0\\
0 & \la & 1 & \cdots & 0\\
\cdots & \cdots & \cdots & \cdots & \cdots\\
0 & 0 & 0 & \cdots & 1\\
0 & 0 & 0 & \cdots & \la
\end{pmatrix}\in M_{k\times k}(\F)$$ $J(\la,k)=\la I+J(0,k)$.\\
Jordan Canonical Form: Matrix formed by Jordan blocks.\\
Admits JCF if exists basis $\be$ s.t. $[T]_{\be}$ is in JCF.\\
Generalized eigenspace $K(\la,T)=\{v\in V\mid \exists k\in\N: (T-\la I)^k v=0\}$.\\
$V=U\oplus K(\la_1,T)\oplus...\oplus K(\la_r,T)$, where $U$ is $T$-invariant with no eigenvalues.\\
$N$ nilpotent if $N^k=0$ for some $k$.\\
Nilpotent $N$ admits JCF with $J(0,k)$.\\

\end{multicols*}



\end{document}